\chapter{Arhitektura i dizajn sustava}
		
		%\textbf{\textit{dio 1. revizije}}\\

		%\textit{ Potrebno je opisati stil arhitekture te identificirati: podsustave, preslikavanje na radnu platformu, spremišta podataka, mrežne protokole, globalni upravljački tok i sklopovsko-programske zahtjeve. Po točkama razraditi i popratiti odgovarajućim skicama:}
	%\begin{itemize}
		%\item 	\textit{izbor arhitekture temeljem principa oblikovanja pokazanih na predavanjima (objasniti zašto ste baš odabrali takvu arhitekturu)}
		%\item 	\textit{organizaciju sustava s najviše razine apstrakcije (npr. klijent-poslužitelj, baza podataka, datotečni sustav, grafičko sučelje)}
		%\item 	\textit{organizaciju aplikacije (npr. slojevi frontend i backend, MVC arhitektura) }		
	%\end{itemize}

	Stil arhitekture koji smo odabrali je arhitektura zasnovana na događajima gdje se događaji javno objavljuju te se pozivaju registrirane procedure, dok komponente koje objavljuju događaj nemaju informaciju koje će sve komponente reagirati i kako. Za razliku od objektno usmjerenog stila, komponente se ne pozivaju eksplicitno, već generiraju signale, tj. događaje. Ova arhitektura je odabrana jer je najefikasnija za obradu korisničkih zahtjeva, laka je za održavanje i reciklabilna za potrebe budućih projekata ili nadogradnje ovog. Što se tiče spremišta podataka, svi podaci će biti spremani te dohvaćani iz univerzalne baze podataka. Mrežni protokoli koji će se pozivati tokom komunikacije klijenta s poslužiteljem su: TCP, IP, HTTP. 
Sustav je organiziran na sljedeći način: 
\begin{figure}[H]
			\includegraphics[width=\textwidth]{slike/arhitekturaSkica.png} %veličina u odnosu na širinu linije
			\caption{grafički prikaz arhitekture}
			\label{fig:arhitektura} %label mora biti drugaciji za svaku sliku
			\end{figure}
Korisnik putem odabranog web preglednika šalje HTTP zahtjev za web aplikacijom i njenim komponentama zadanom web poslužitelju. Aplikacija komunicira s bazom podataka u kojoj su sadržani svi podatci i iz nje izvlači sve potrebne podatke za klijentov zahtjev. Poslužitelj, kada aplikacija dohvati sve potrebne podatke, odgovara na klijentov zahtjev sa statusom 200 OK (ako je sve u redu) i šalje mu HTML dokument koji se prikazuje u zadanom web pregledniku.
Programski jezik koji smo koristili za izradu web aplikacije je Java za backend te Javascript za frontend dio.
Arhitektura sustava se temelji na MVC konceptu, stilističkoj varijaciji arhitekture zasnovanoj na događajima. Odabrali smo taj koncept jer odvaja korisničko sučelje od ostatka sustava, što čini razvoj i nadogradnju komponenata jednostavnijima. Sastoji se od modela, pogleda i upravitelja.
		
\begin{itemize}
\item Model – dojavljuje sebi pridruženim pogledima i upravitelju kada je došlo do promjene u njegovom stanju. Ove dojave omogućuju pogledu da prikaže obnovljeno stanje modela, a upravitelju promjenu dostupnog skupa naredbi
\item View (Pogled) - od modela dobija informacije koje su mu potrebne za prikaz korisniku
\item Controller (Upravitelj)  – šalje naloge modelu koji ažurira svoje stanje i naredbe pogledima kojima mijenja prikaz modela
\end{itemize}
		

				
		\section{Baza podataka}
			
			%\textbf{\textit{dio 1. revizije}}\\
			
		%\textit{Potrebno je opisati koju vrstu i implementaciju baze podataka ste odabrali, glavne komponente od kojih se sastoji i slično.}
		
			\subsection{Opis tablica}
			

				%\textit{Svaku tablicu je potrebno opisati po zadanom predlošku. Lijevo se nalazi točno ime varijable u bazi podataka, u sredini se nalazi tip podataka, a desno se nalazi opis varijable. Svjetlozelenom bojom označite primarni ključ. Svjetlo plavom označite strani ključ}
				
				
				%\begin{longtblr}[
				%	label=none,
				%	entry=none
				%	]{
				%		width = \textwidth,
				%		colspec={|X[6,l]|X[6, l]|X[20, l]|}, 
				%		rowhead = 1,
				%	} %definicija širine tablice, širine stupaca, poravnanje i broja redaka naslova tablice
				%	\hline \multicolumn{3}{|c|}{\textbf{korisnik - ime tablice}}	 \\ \hline[3pt]
				%	\SetCell{LightGreen}IDKorisnik & INT	&  	Lorem ipsum dolor sit amet, consectetur adipiscing elit, sed do eiusmod  	\\ \hline
				%	korisnickoIme	& VARCHAR &   	\\ \hline 
				%	email & VARCHAR &   \\ \hline 
				%	ime & VARCHAR	&  		\\ \hline 
				%	\SetCell{LightBlue} primjer	& VARCHAR &   	\\ \hline 
				%\end{longtblr}
				
				
			
			\subsection{Dijagram baze podataka}
				\textit{ U ovom potpoglavlju potrebno je umetnuti dijagram baze podataka. Primarni i strani ključevi moraju biti označeni, a tablice povezane. Bazu podataka je potrebno normalizirati. Podsjetite se kolegija "Baze podataka".}
			
			\eject
			
			
		\section{Dijagram razreda}
		
			%\textit{Potrebno je priložiti dijagram razreda s pripadajućim opisom. Zbog preglednosti je moguće dijagram razlomiti na više njih, ali moraju biti grupirani prema sličnim razinama apstrakcije i srodnim funkcionalnostima.}\\
			\textbf{UserModel} služi za spremanje podataka o korisniku te za autentifikaciju i autorizaciju korisnika. 
			\textbf{UserModel} generalizira \textbf{Admin} i \textbf{StoreModel}.
			 
			\textbf{PrivacyModel} pohranjuje podatke o tome koji će podaci o korisniku biti javni, a koji privatni.
			  
			\textbf{Admin} sadrži metode koje može izvršiti isključivo korisnik s admin pravom pristupa(odobravanje promjena cijena, zabrana pristupa drugim korisnicima...) 
			
			\textbf{StoreModel} modelira trgovinu. 
			
			\textbf{ProductModel} modelira proizvod. Proizvod je jednoznačno određen barkodom kako bi više trgovina moglo imati isti proizvod u ponudi. 
			
			\textbf{PriceChangeLog} služi za opisivanje promjena cijena proizvoda u nekoj trgovini u nekom vremenskom razdoblju. 
			
			\textbf{PriceChangeRequestModel} modelira zahtjev za promjenom cijene koje šalje korisnik za proizvod u nekoj trgovini u kojoj se stvarna cijena razlikuje od cijene navedene u aplikaciji.  
			
			\textbf{NotificationModel} modelira obavijesti koje admin može slati korisniku ili trgovini.
Klase iz data access paketa sadržavaju isključivo statičke metode koje služe za komunikaciju s bazom podataka.

\begin{figure}[H]
			\includegraphics[width=\textwidth]{slike/dijagramRazreda.png} %veličina u odnosu na širinu linije
			\caption{dijagram razreda}
			\label{fig:dijagramRazreda} %label mora biti drugaciji za svaku sliku
			\end{figure}
			
			%\textbf{\textit{dio 1. revizije}}\\
			
			%\textit{Prilikom prve predaje projekta, potrebno je priložiti potpuno razrađen dijagram razreda vezan uz \textbf{generičku funkcionalnost} sustava. Ostale funkcionalnosti trebaju biti idejno razrađene u dijagramu sa sljedećim komponentama: nazivi razreda, nazivi metoda i vrste pristupa metodama (npr. javni, zaštićeni), nazivi atributa razreda, veze i odnosi između razreda.}\\
			
			%\textbf{\textit{dio 2. revizije}}\\			
			
			%\textit{Prilikom druge predaje projekta dijagram razreda i opisi moraju odgovarati stvarnom stanju implementacije}
			
			
			
			\eject
		
		%\section{Dijagram stanja}
			
			
			%\textbf{\textit{dio 2. revizije}}\\
			
			%\textit{Potrebno je priložiti dijagram stanja i opisati ga. Dovoljan je jedan dijagram stanja koji prikazuje \textbf{značajan dio funkcionalnosti} sustava. Na primjer, stanja korisničkog sučelja i tijek korištenja neke ključne funkcionalnosti jesu značajan dio sustava, a registracija i prijava nisu. }
			
			
			%\eject 
		
		%\section{Dijagram aktivnosti}
			
			%\textbf{\textit{dio 2. revizije}}\\
			
			% \textit{Potrebno je priložiti dijagram aktivnosti s pripadajućim opisom. Dijagram aktivnosti treba prikazivati značajan dio sustava.}
			
		%	\eject
		%\section{Dijagram komponenti}
		
			%\textbf{\textit{dio 2. revizije}}\\
		
			% \textit{Potrebno je priložiti dijagram komponenti s pripadajućim opisom. Dijagram komponenti treba prikazivati strukturu cijele aplikacije.}