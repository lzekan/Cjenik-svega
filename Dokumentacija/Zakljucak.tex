\chapter{Zaključak i budući rad}
		
		%\textbf{\textit{dio 2. revizije}}\\
		
		 %\textit{U ovom poglavlju potrebno je napisati osvrt na vrijeme izrade projektnog zadatka, koji su tehnički izazovi prepoznati, jesu li riješeni ili kako bi mogli biti riješeni, koja su znanja stečena pri izradi projekta, koja bi znanja bila posebno potrebna za brže i kvalitetnije ostvarenje projekta i koje bi bile perspektive za nastavak rada u projektnoj grupi.}
		
		 %\textit{Potrebno je točno popisati funkcionalnosti koje nisu implementirane u ostvarenoj aplikaciji.}
		
		
		Zadatak projekta "CjenikSvega" je bio razviti web-aplikaciju koja omogućuje korisnicima da pronađu cijene proizvoda u različitim trgovinama te da slijede i ažuriranje cijena putem slanja slika. 
		Okupljanje tima i dodjela projektnog zadatka omogućili su da se postavi jasna vizija projekta te da se svi tim članovi usaglase u pogledu ciljeva i zahtjeva projekta. Dokumentiranje zahtjeva je također važno, jer je to osnova za daljnji rad na projektu te omogućava da se razumiju zahtjevi korisnika i kako se aplikacija treba ponašati. Tijekom dokumentiranja, tim se susreo sa izradom UML dijagrama obrazaca uporabe, sekvencijskih dijagrama, dijagrama razreda, dijagrama aktivnosti, komponenti, razmještaja i stanja.
		
		U procesu rada na ovom projektu te ponajviše u drugom ciklusu, tim je naučio korištenje Express.js framework-a rame uz ramena sa drugim tehnologijama kao što su JavaScript, HTML i CSS te se susreo i riješio problem s radom s bazama podataka. Iako nam je tehnologija bila poznata od prije, implementacija nije došla sama od sebe pa smo tako morali uložiti podosta vremena na svladavanje tehnologije. Glavni izazov je bio upravo rad u timu, kako rasporediti posao tako da napredak ide nesmetano. Iskustvo igra veliku ulogu pa kako je ovo bio među prvim projektima članova ovog tima, sigurno je da bi idući projekt puno brže tekao.

Neke od nadogradnji na sustav bilo bi svakako izrada mobilne aplikacije da se olakša slanje slika. Korištenjem tehnologije OCR za automatizirano prepoznavanje cijena na slikama pomoglo bi se administratoru pri validaciji zahtjeva promjene cijene.

Zaključno, ova web-aplikacija ima potencijal da bude koristan alat za potrošače i vlasnike trgovina. Omogućujući korisnicima da jednostavno prijave netočne cijene putem kamere pametnog telefona, aplikacija može pomoći osigurati da su cijene u trgovinama točne i ažurne. Osim toga, mogućnost administratora trgovine da brzo ispravi sve netočnosti može pomoći u izgradnji povjerenja kod kupaca i održavanju poštenih cijena. Iako je ovaj projekt bio samo prototip, on pokazuje potencijal za sveobuhvatniju verziju koja bi se mogla implementirati u budućnosti.
		\eject 